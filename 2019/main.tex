\documentclass[uplatex,a4j,11pt]{jsarticle}
\renewcommand{\abstractname}{要旨}
\renewcommand\baselinestretch{0.8}
\usepackage[vdivide={3cm,,3cm}]{geometry}
\usepackage{fixltx2e}
\usepackage{setspace}
\usepackage{wrapfig}
\usepackage{setspace}
\usepackage{indent}
\usepackage{fancybox}
\usepackage{type1ec} %T2Aなどを使う.
\usepackage{textcomp} % T1のシンボルを使う
\usepackage[T2A,T1]{fontenc}%フォントでT2AとT1を使う
\usepackage[utf8]{inputenc}% ファイルがUTF8であること
\usepackage{zi4}%等幅フォントをInconsolataで
\usepackage[multi,deluxe,uplatex,jis2004]{otf}%繁簡ハングル多ウェイトOpenTypeなし
\usepackage[prefernoncjk]{pxcjkcat} % なるべく「半角」扱いで.
\cjkcategory{sym18}{cjk} % sym18 (U+25A0 - U+25FF Geometric Shapes) を和文文字あつかい
\usepackage[caption=main,english,french,german,russian,japanese]{pxbabel} %基底言語は日本語
\usepackage{plext} %傍点をふる
\usepackage{pxrubrica} %ルビをつける
\usepackage{comment} %コメント環境の用意
\usepackage{siunitx} %% SI単位系の出力
\usepackage{ascmac} %ボックス
\setcounter{tocdepth}{3}
\renewcommand{\thefootnote}{\arabic{footnote}}
\usepackage[dvipdfmx]{graphicx}
\newcommand*{\tabref}[1]{表~\ref{tab:#1}} %表のカウント
\newcommand*{\figref}[1]{図~\ref{fig:#1}} %図のカウント
\newenvironment{mytab}[3][htbp] %表を中央揃え
 {\begin{table}[#1]\begin{center}\caption{#2}\label{#3}}
 {\end{center}\end{table}}
\newcommand{\myfig}[4][width=.8\linewidth]{ %図を中央揃え
\begin{figure}[htbp]
   \centering\includegraphics[#1]{#2}
   \caption{#3}\label{fig:#4}
\end{figure}}
\title{田中小実昌「ポロポロ」における\\鋳型としての「人魂」について\footnote{発表日は2017/7/17。}}
\author{佐藤正尚\footnote{言語情報科学専攻博士後期課程1年(31-187007) masasato@phiz.c.u-tokyo.ac.jp}}
\date{2018/7/31}

% prevent hyphenation
\hyphenpenalty=10000\relax
\exhyphenpenalty=10000\relax
\sloppy

\begin{document}
\maketitle
%\thispagestyle{empty}
\tableofcontents
\section{田中小実昌と「ポロポロ」}
田中小実昌の半生が色濃く反映され、一種の私小説のようになっている「ポロポロ」を論じるにあたって田中小実昌についての伝記的事実を確認しておく。

大正14年(1925年)4月29日に東京府豊多摩郡千駄ヶ谷の東京市民協会(現東京都世田谷区代田 東京都民教会 )で牧師をしていた田中種助の子として生まれる。三島由紀夫も同年生まれている。4歳の時に小倉に移住し、その後、当時の小学1年生の頃に呉に移住した。これは、父種助が小倉のシオン山教会で牧師をつとめた後、呉教会から牧師として招聘されたためであった。呉一中(現呉三津田高)に落第して福岡の西南学院中等部に1年間在籍。2年の編入試験で呉一中に戻った。旧制高校は福岡に入った。その前後で、東三津田の自宅で洗礼を受ける。昭和19年12月に19歳で山口の連隊へ入営。湖北省と湖南省の県境にある粤漢鉄道警備部隊に編入され、行軍の日々を送るなかで、アメーバ赤痢・マラリヤ・コレラにかかる。短編集『ポロポロ』の連作ではその日々のことが中心に描かれている。昭和21年8月に復員。22年4月から種助の手続きで東京大学哲学科に復学したものの、同年4月から渋谷東急4階にあったストリップ劇場『東京フォーリーズ』で働き始めた\footnote{ただし、現在イメージされるストリップ劇場とは大きく異なる。この頃のストリップはもともと活人画に由来したもので、舞台演出などを芸術活動の一つとして活動している者もいた。ただし、戦後初のカラー映画『カルメン故郷に帰る』(1951)で、芸術家を志した結果ストリッパーになって帰郷するカルメンことおきんが、その愚かさを強調された主人公となっていたことからわかるように、当時すでに揶揄の対象にもなっていた。また、大学に通わなかったのは小実昌の生活態度に起因するところもあるが、40年代末から50年代末の頃は、大卒だからといって働き先が明確に決まるわけではなかったこともその理由として考えられる。以下を参照のこと。堀江敏幸・池内紀「愛に近い異様なもの」『ユリイカ 総特集田中小実昌の世界 6月臨時号』、第32巻第9号(通巻434号)、青土社、2000年、39頁。}。その後、GHQの将校クラブなどで働いた。クラブから解雇された後、テキ屋の子分になって福井県に遠征したこともある。その後、東京に戻ると、将校クラブで働いていた時に英語を身につけた経験を活かして、昭和31年から『エラリー・クイーンズミステリマガジン』で翻訳を手がけるようになる\footnote{最初の翻訳作品は以下を参照のこと。ジェイムズ・M・ケイン「冷蔵庫の中の赤ん坊」、『エラリイ・クイーンズ・ミステリ・マガジン』、田中小実昌訳、11月号(01巻05号 通巻005号)、1956年。}。1970年代に入ってから小説を単発的に発表し、79年(昭和54年)に直木賞を『香具師の旅』(1972年2月)にて受賞し、同年に谷崎潤一郎賞を「ポロポロ」(『ポロポロ』(1979年5月))にて受賞した。

小実昌は直木賞と谷崎賞を受賞した後、80年代に入ると哲学書を大量に引用する小説を書くようになる。それらは『カント節』としてまとめられる。プラトン、カント、スピノザ、ベルクソンの言葉を引用しながら小説を書いていくスタイルになっていった。その後、テレビ番組やCMに出演するようになった。晩年は糖尿病を患い、2000年2月26日にロサンゼルスにて客死 \footnote{来歴については下記を参照のこと。「酔いの明星 田中小実昌氏と重兼芳子さんの``受賞のヒミツ''」『週刊朝日』、8月第3号、朝日新聞出版、1979年、165-166頁。田中小実昌『アメン父』、河出書房新社、1989年。「田中小実昌年譜」『昭和文学全集』、第31巻、小学館、1988年、1001-1004頁。『ユリイカ 総特集田中小実昌の世界 6月臨時号』、第32巻第9号(通巻434号)、青土社、2000年。}

本論で取り上げる「ポロポロ」は主に彼の10代前半の出来事が参照されていると考えられる。「ポロポロの作品分析に入る前に、先行研究で田中の小説家としての評価の全体をまとめる。


\section{田中小実昌に対する評価}

まず、谷崎潤一郎賞における評価を確認する。候補作品には三木卓『野いばらの衣』、山崎正和の戯曲『地底の鳥』、李恢成『見果てぬ夢』の評価も同様に高い。円地文子、丸谷才一、大江健三郎はそれぞれこれらのうちの1つを推薦している。しかし、大勢として小実昌の授賞でまとまっていた。

遠藤周作は、「『ポロポロ』はいかに正確に自らを語られるかと言う[原文ママ]自己検証の作品であり、その正確に執拗に迫るくりかえしがこの作品に迫力を与えている\footnote{『中央公論』、第94年第11月号、中央公論社、1979年、353頁。}」と述べている。大岡昇平は「寝台の穴」にある一節を引きつつ、「このような文学の本質に触れた言説が、物語の中に混じっている\footnote{前掲書、354頁。}」と評価している。丹羽文雄は、「ポロポロ」に「宗教的感動\footnote{前掲書、355頁。}」を覚え、「この小説は日本文学に新しい一頁を加えた\footnote{同書。}」と激賞している。吉行淳之介は「大尾のこと」を取り上げ、そこでのひとの終始を描くことができるのは神だけであるという内容の一節を取り上げながら、「文学上のポロポロ\footnote{前掲書、356頁}」として小実昌の連作を総括している。

次に、田中小実昌についての先行研究をみていく。小実昌の1990年以降の著者目録を自身で編集しているなど小実昌について精力的に分析している\footnote{伊藤義孝「田中小実昌著書目録~---~1990年~2001年」『愛知淑徳大学国語国文』、第25巻、2002年、1-18頁。}伊藤義孝は、後期の軍隊連作小説のプロットと描写の分析から次のように述べている。

\begin{quote}
  つまり、小説を書くという行為にせよ、発話行為にせよ、外界に対し、自己の内面をアウトプットする行為自体が意識の屈折をはらみ、誤謬、隠蔽、欺購、更に自己欺購にさえ通じる可能性を孕んでいることに小実昌は気づいているのだ。だからこそ、小実昌は事実を伝達すること自体を放棄し、「カンケイない」「仕方がない」「しょうがない」といった言葉でわれわれを突き放してしまうのだ\footnote{伊藤義孝「田中小実昌論~---~その語りのあり方」『愛知淑徳大学国語国文』、第28巻、2005年、185頁。}。
\end{quote}

また、この「事実を伝達すること自体を放棄」する文体が戦後文学史にどのように組み入れらかについて次のように述べている。

\begin{quote}
  さて、田中の文学史的な位置づけについては、これは序でも述べたごとく、現在全くと言っていいほどなされていない。しかし、彼の従軍体験や、それにまつわる復員体験を描いた一連の小説は、現在の文学史上における「戦争文学」のあり方を再考させる大きなきっかけとなるに違いない。とりわけ、「内向の世代」と共通する思考のあり方は、田中小実昌の文学を新たな枠組みに組み込むことが可能なのではないかということを示唆してくれる\footnote{前掲書、186頁。}。
\end{quote}

以上のように、伊藤は他の論者と同様に、小実昌が『ポロポロ』以後、事実を描くために物語を用いざるをえない自覚について、「意識の屈折」や「自己欺瞞」といった言葉で説明し、読者を「突き放」すものだとしている。しかし、それは批判ではなく、内向の世代と比較することで、戦争文学のジャンルの拡張可能性を指摘している。

志賀\ruby[g]{浪幸子}{さちこ}も同じような議論を立てている。
\begin{quote}
  (略)「物語」への不信はこの作品[=「北川は僕に」]をはじめとして\textgt{「海」に昭和五十三年三月より発表された一連の「兵隊もの」の中で、繰り返し描かれている}。(中略)\textgt{「ポロポロ」が「受ける」ことしか出来ない、所有することも出来ないものであったのは、それが外部からの意味づけや限定を拒みつづけるものであったからにほかならない}。それは意味のある「言葉」を敢えて付すことの出来ないものであったのであり、その必要の無いものであった。だからこそ、「ポロポロ」は純粋な神への賛美たり得たとも言える。こうした環境に育った田中が外部からの安易な意味付け、「言葉」への過信に不信の目を向けたの必然であった。(中略)\textgt{「物語」にしてしまう、という事は、どういうことなのか。それは事実を「対象化」する事で、実際のものとは別のものを創り出し、別の意味を付加してしまうことと言っていいかもしれない}(引用強調は筆者)\footnote{志賀浪幸子「田中小実昌「ポロポロ」」『私小説研究』、第4号、法政大学大学院私小説研究会、2003年、60-61頁。}。
\end{quote}

伊藤が小実昌が「事実」を放棄したとしているのに対して、志賀はさらに議論を進めて「田中が外部からの安易な意味付け、「言葉」への過信に不信の目を向けた」というように、「ポロポロ」というフレーズこそ言葉の放棄であると述べている。

こうした「事実」や「言葉」の放棄を、語り手が登場人物たちに介入しない観察者に徹していると別の視点から述べているのは、保坂和志である。

\begin{quote}
  でも、荒っぽく言うと、\textgt{小実昌さんは子どものように「見る人」「見てばかりいる人」}\textgt{だった}。いままで引用した箇所でも見てばかりいる(引用強調は筆者)\footnote{保坂和志「小実昌さんのこと」『新潮』、第95巻第5号、新潮社、2000年、137頁。}。
\end{quote}

「この見てばかりいる人」は、「ポロポロ」が幽霊の登場によってある奇妙な出来事の解釈が決着させられるという展開にも関係している。

\begin{quote}
  しかしやっぱりどうしても見ていられなくて、小実昌さんは途中で出てきてしまった。じかに本人の口からも聞いたことだが、\textgt{ホラー映画はどうしてもダメということだった}。引用した箇所の妹に訊くところで、「わりとしつこく」と書いていることに注目してほしい(引用強調は筆者)\footnote{前掲書、147頁。}。
\end{quote}

のちに、こうした小実昌のホラーへの感性に注目した議論をする際に保坂の引用に再度注目する。

保坂と同様別の小説家も長い評論を執筆しており、こうした田中小実昌論をとは異なった視点を提示しているのが堀江敏幸である。堀江は『ポロポロ』の中でもとりわけ「北川はぼくに」は、物語に収束しない出来事をどのように表現するかの意識が現れていると指摘している\footnote{堀江敏幸「フィリップ・マーロウを訪ねたチェスの名人」『書かれる手』、平凡社、2000年、149頁。}。また、田中は「ポロポロ」以降の物語を語ることを批判する女性の登場人物を描くようになるが、実は、それ以前の『香具師の旅』に収められている「ミミ」で描かれる聴覚障害者の売春婦「ミミ」とのコミュニケーションに、そうした批判に対する応答としてあらかじめ書き込まれていたという可能性を見出している。

最後に、取り上げる先行研究はキリスト教に関するものである。「ポロポロ」は、独立教会の牧師を務めていた父親が物語の中核を成しているが、そうしたキリスト教的背景に注目している論者として、富岡幸一郎があげられる。『ポロポロ』の後に執筆された『アメン父』(1988)を中心に、後期小実昌小実昌の思想を解明しようとしている。主に、カール・バルト『ローマ書』や「過去と未来~---~ナウマンとブルームハルト」から、宗教的可能性に対する批判が小実昌にもあったとしている。宗教的可能性はカール・バルトの言葉で、神を理性によって捉えようとすることを意味しており、その可能性が信仰だけでなく、広くイデオロギー(マルクス主義、実存主義、無神論)をも生んだとされる。というのも、バルトにとって、それらはすべて神の業について理解しようとする思想だからである。では、どうしてこうした意味での宗教が批判されねばならないかというと、内的な世界に閉じこもり現実の生活から遠ざかってしまうからである。イエスは社会に積極的に参画していくという意味で活動的であり、実際、信仰とは実生活の中における事実である。バルトは、こうした考えがブルームハルトのいうところの「宗教ではない! 神の国だ!」の意味であるとする。また、『アメン父』で小実昌が洗礼について書いている箇所を引用して、「これは田中小実昌が、父の信仰とその生涯から学びとり、そして\textgt{自身でも体験したキリスト信仰の確信のようなものだろう}(引用強調は筆者)\footnote{富岡幸一郎「祈りの言葉のリレー~---~田中小実昌論」『文芸評論集』、アーツアンドクラフツ、2005年、168頁。}」と述べている。

以上の先行研究をまとめると、田中小実昌の短編集『ポロポロ』は「北川はぼくに」を中心に、小説において物語ることが現実の出来事を語ること手段ではないことに疑念を持ちつつも、「ポロポロ」の宗教的経験などのアプローチから、小説を書くことのアプローチを可能としている。それはそれまでの戦争文学とは異なるので、ジャンルの拡張を示唆する。また、哲学小説と呼ばれるような一連の作品は、「ポロポロ」の問題系を引き継ぎ、『アメン父』では小実昌が父の思想を探ることを通じてその思想の核心となっているブルームハルト的な信仰に理解を示すようになったと考えられる。

\section{「ポロポロ」の背景}

小実昌の父種助の活動は日本バプテスト西部組合や西南学院のボールデン院長解任事件など多くの問題を引き起こしたことが知られており\footnote{枝光泉「日本バプテスト西部教会の歴史~---~「アサ会」事件について」『キリスト教社会問題研究』、第48号、同志社大学人文科学研究所キリスト教社会問題研究会、1999年、102-124頁。金丸英子「西南学院とアサ会~---~ボールデン院長の解任を巡って」『西南学院史紀要』、第6号、西南学院、2015年、49-59頁。}、「ポロポロ」で表現されているように単に嫌がられていたというよりも西日本のバプテスト系キリスト共同体から破門されているように、キリスト教共同体からも排除されていた。

小実昌が、『カント節』(福武書店、1985年)以降に『ポロポロ』のテーマに回帰したのは、『アメン父』だった。同書では、「ポロポロ」よりも詳細な家族史や集会所について語られている。例えば、呉の灰ヶ峰を背にして右手にある三津田の家や集会所の位置関係も正確に知ることができる。「ポロポロ」の作中でも出てくる山の側面にある平屋が一番下にあり、種助が「中段」と呼んだ旧阪田(『アメン父』では坂田)別荘の集会所が、上の山の尾根にある家との間にある\footnote{田中小実昌『アメン父』、河出書房新社、1989年、5頁。}。集会所の周りにはドングリや桜があった。「ポロポロ」では戦時中に参拝者が少なかったことから一番下の家を集会所代わりにしているが、『アメン父』の記述をみていくと、最上段の家の離れの扱いだったと思われる。『アメン父』では多くの草花や果物の話が登場するが、「ポロポロ」で赤大根と対になって登場する夏ミカンに関する言及は一度もない。ただし、静岡県にあった田中家の墓には墓地にはミカンの木が生えているということが伝聞として記されている。

こうした「ポロポロ」を形成する土地である広島県の軍港町出会った呉は、バプテスト派の教会が明治に入ってからできており、実業家が入信する程度には街で知られていた。「ポロポロ」では父が特高に尋問されている様子が描かれているのも、呉が。『アメン父』では、ミッドウェー海戦の後に特高が家に尋問にやってきたことが触れられている。特高が自分の家に来て普段はしないような細かなことまで尋問していった後で、種助はミッドウェー海戦が大敗だったことを悟っていたという\footnote{前掲書、25頁。}。

ここで、一旦、父についての来歴を簡単にまとめる。小実昌の父は本名を田中種助といい、明治18年に生まれた。明治41年(1908)年5月、渡米し、農園で働いていた。明治45年2月、縁あって身を寄せることになったシアトルのユニテリアン系日本人組合教会にて\ruby[g]{久布白}{くぶしろ}直勝による洗礼を受ける。その後、アメリカで神学を学び、日本でも宗派を変えてバプテスト派の神学校に通った。大正9年(1920)、落城した東京市民教会で亡くなった久布白の代わりに牧師に就任し、その後、九州小倉のシオン教会の牧師になるものの、すぐに辞して八幡製鉄関係者の所持する山の山番をした。昭和2年(1927)、アサ会の設立と、遵聖への改名を決意をすると、広島県呉市本通の呉バプテスト教会の牧師が欠員したためにこの職を預かることになった。それが、昭和4年(1929)のことであった。昭和6年(1931)、実質的にアサ会となっていた呉バプテスト教会が西部組合と分裂、1920年より回心していたセーラー万年筆阪田久五郎から教会が買い上げた別荘を集会所としていた。終戦後、10年ほど生き延びて昭和33年(1958)に没した。次に、「ポロポロ」で語り手の前に現れた父方の祖父について紹介する。

父方の祖父の名前は田中定平といった。種助は田中家の養子に来たので、定平は実父ではない。妻(種助の養母)の名前は『アメン父」からは知ることができない。田中家はもともと富士川あたりの土地の郷主であり、定平はその息子だったが、種助の祖父が夭折したあと、しばらくしてから没落した。借金を抱えており、種助は村に残っていた土地を処分することで清算した。 

次に、実父について紹介する。種助の実父は山本仁右衛門といった。旅館、碁会所、薬湯屋などの経営していた。種助を産んだ妻(実母)は、成田トク。種助を産んだ後、山本と離婚し、成田家に入った。

「ポロポロ」は語り手が死んだはずの祖父を見た話となっているが、『アメン父』にも、東玉川に住んでいたころ、田園調布に住んでいた霊媒師の女性に出会った話が登場する。しかし、小実昌は、自分の父が霊媒は宗教ではないと考えたはずだと想像している\footnote{前掲書、103頁。}。

家族の来歴について触れたので、次に、小実昌が「ポロポロ」という言葉の由来について語っている個所に言及する。作中では、「ポロポロ」という言葉はパウロがつまったものだとされている。しかし、これが部分的に創作であったことが平岡篤頼との対談で小実昌自身が証言している。平岡がアサ会の信者の集会について尋ねて、それがある種の神の賛美だったと小実昌は次のように語っている。

\leftskip=2zw
  \hangindent=4zw \textgt{ 平岡} それを「ポロポロ」と呼んでたんですか。

  \hangindent=4zw \textgt{ 田中}  いやいや、それは僕が「ポロポロ」と短編に題名をつけただけでありまして。というのは、あるおじさんが、ハンコ屋さんなんですけど、ポロポロ、ポロポロ言うんですよ、一人一人言うことは違うんでね。それでそれを短編の題名にもってきただけですよ。``ポロポロ''は、``南無妙法蓮華経''とか``南無阿彌陀仏''とかに思わたんじゃ困るんでしてね、ただ``ポロポロ''言ってる、ただ何か言ってるということだけのことなんですよ。

  \hangindent=4zw \textgt{ 平岡} 要するに、``ポロポロ''と言うのは意味がなくて脈絡もなくて、ただ``ポロポロ''出てくると言うことですね。それは``ポロポロ''と訴えかける、祈りかけるのか、それとも神がかりみたいにひとりでに出てくるのか。

  \hangindent=4zw \textgt{ 田中} ひとりででしょうね。ひとりでといっても、これはひとりでに出たものであるか、自己催眠であるか分析すれば色々あると思いますよ。ただポロポロ、ポロポロとやってて、ポロポロがとまらないなんていう人もいて、それだって精神分析をすれば、やっぱりデモンストレーションの一種だとかいろんなことを言うかも知れないけど、そんなことを言ったってしょうがないんでね。ただポロポロ、ポロポロ、あんまり意識的ではないんでね、お題目を何遍唱えれば何とかなるというんじゃないんですね\footnote{田中小実昌・平岡篤頼「文学的ポロポロ~---~早稲田文学対談11」『早稲田文学 [第8次]』、4月号、1980年、11頁。}。

\leftskip=0zw

対談ではその後、小説「ポロポロ」の語り手と同じく、田中はポロポロと言うことはなかったが、小実昌自身は「ギャア」といった声を出したという。それについて、田中はウィリアム・ジェイムズの『宗教経験の諸相』を引きながら、ポロポロとは別の形で宗教経験があったことを語っている\footnote{前掲書、12頁。}。「ポロポロ」と宗教的経験で扱われる内容がさらに深く掘り下げられる『アメン父』を執筆している頃の田中は父の説教がどのようなものであったのかを積極的に引用するようになる。例えば、次のような説教である。

\begin{quote}
  ところが、アサの十字架は生きている。わたしたちが十字架にあうということは、じつは、罪人としてあっている。もうすでになくなって、帳消しにされたものを信ずるのではない。ところが、わたしがならったキリスト教は、罪は帳消しになってるんだから今さら、救われるとか、救われないとか、そんな問題はない、もう救われてる、だから、それを信じなさい、というものだった。/しかし、わたしたちは事実の世界からいかねばならない。信念の世界ではない。信念からはそういう信念をもつことができる。また、もたされるんです。だが、そうではない。わたしたちはここにこうしてアサを受けてわかることは、たえずさからっていて、たえず神を拒んでいる自分を見る\footnote{田中小実昌「十字架」『ユリイカ 総特集田中小実昌の世界 6月臨時号』、第32巻第9号(通巻434号)、青土社、2000年、63頁。}。
\end{quote}

キリスト教者である遵聖は「わたしたちは事実の世界からいかねばならない」という聖書に描かれている物語から実生活において「たえず神を拒んでいる自分を見る」ような「アサを受け」るような経験が重要だと説いている。この「事実」という言葉は、『ポロポロ』以後に小実昌が対談などでことあるごとに言及するようになっていた。例えば、井上忠との対談で小実昌はしきりに事実を問題にしていた\footnote{井上忠・田中小実昌「宗教--その「根拠」を問い直す」『季刊 仏教』、第5号、1988年、12-31頁。}。「ポロポロ」は独立教会の牧師をしている父という明らかに事実を反映している物語である一方で、物語ることを小実昌が警戒しているのは、宗教的経験と無縁ではない。

さらに、この対談は小実昌と井上が「宗教はココロの問題ではない。つまりは、宗教は宗教心とは関係とも関係はない\footnote{前掲書、35頁。}」とまとめられるフレーズに強く賛同している点が重要である。『アメン父』発売後、田中小実昌と富岡幸一郎は対談をしている、その中で、小実昌は次のように述べている。

\begin{quote}
  私たちは物語でかためられていて、何やったって物語なんだから。聖書も聖書物語になるし、イエスもイエス物語になるし、神様も神様物語になっちゃうし。だけど、そうじゃないと思うんだな。例えば、聖霊なんてのは、物語じゃないと思うんですよ\footnote{田中小実昌・富岡幸一郎「田中小実昌と「アメン父」」『すばる』、6月号、集英社、1989年、270頁。}。
\end{quote}

「ポロポロ」でもゲッセマネのイエスが神に祈りを捧げた時、それが果たして聖書に書かれている言葉を発言していたのかどうかを訝しる場面がある\footnote{田中小実昌「ポロポロ」、前掲書、362頁。}。のちに再度議論することになるが、小実昌が「ポロポロ」で示しているのは、先行研究で言われているように事実や言葉からの撤退ではない。むしろ、事実と物語の距離をどのように表現するかという問題なのである。

このことは、そもそも、小実昌のキャリアは翻訳から始まっていることからも裏付けられるだろう。翻訳とはまさにある言語の事実を別の言語の事実として移し替える作業である。さらに、こうした事実と物語の間の溝への警戒を抱いていた時期の後で、小実昌は広辞苑改訂版の座談会に出席している\footnote{田中小実昌・森本哲郎・山口明穂「座談会 言葉の苑」『図書』、第594号 岩波書店、1998年、2-13頁。}。この中で、小実昌は出席者とともに言葉の集積打つそのものである辞書について実に雄弁に語っている。ここで、言葉について意識的でありすぎるためにかえってその限界を理解しているといった逆説的なテーマを見出す反論もありえるだろうが、これらの小実昌の意見を踏まえて、『アメン父』をめぐる富岡との対談で小実昌自身がこうした考えを否定するかのような見解を持っている。この対談で、小実昌は「ポロポロ」が物語にすぎないことを強調している。

\leftskip=2zw


  \hangindent=4zw  \textgt{ 富岡} 書き方については『ポロポロ』を書かれたころと、これ[=『アメン父』]を書かれた時と、ご自分ではどうですか。

  \hangindent=4zw \textgt{ 田中} 父に関しては、変わらないですね。だけど、自分の理解の仕方が違えば、父の像も違ってくるというふうに、普通はなるんだけど、そういうことともちょっと違うんだな。

 \hangindent=4zw \textgt{ 富岡} 年を取ればとか、経験を積めばということも……。

 \hangindent=4zw  \textgt{ 田中} まるっきり違うことですからね。それが大変困るんですね。そういう困っていることを今度は書いているわけで、『ポロポロ』の時は、困っていることをあまり書いていませんからね。違いといえば、そういう違いというか。

 \hangindent=4zw \textgt{ 富岡} 『ポロポロ』のほうが……。

 \hangindent=4zw \textgt{ 田中} いわゆる小説的ですね。

 \hangindent=4zw \textgt{ 富岡} ええ、ちょっと客観的で、小説的。石段のところにおじいさんの影が見えたとか。

 \hangindent=4zw \textgt{ 田中} ああいうのは、いくらか話が入っているわけですからね。まあ今度のは、これは困っていることも何もかも書こうと思って書いたから、体裁をつくってないですよ。前よりかね。(中略)今まで通りの書き方だと、物語がウソとは言いませんけど、すぐ物語になっちゃうし、それじゃ違うんじゃないかと。ウソになっちゃ、もちろんいけないしということでね\footnote{前掲書、274頁。}。 

\leftskip=0zw

富岡が「ポロポロ」で死んだ祖父と出会ったことについて言及すると、「ああいうのは、いくらか話が入っているわけですからね」と「ポロポロ」の虚構性を認めつつも、そうした「今まで通りの書き方だと、物語がウソとは言いませんけど、すぐ物語になっちゃうし、それじゃ違うんじゃないかと。ウソになっちゃ、もちろんいけないしということでね」といったように倫理的に物語は虚偽ではあってはならないが、物語は決して事実ではないということを認めるような発言をしている。ここで問題になっているのは、言葉や事実ではなくて、むしろ物語という形式によって語ること自体なのだ。

こうした点から、「ポロポロ」という異言によって表現することは、先行研究や谷崎賞での評論で繰り返し述べられている言葉の芸術であるところの小説の限界を露呈しているのではない。小実昌が言葉を使用して事実を語ることが物語になることについて、どのようにアプローチしているのか、そして、そのアプローチによってどのような物語が作られているのか、という点こそ検討されるべきことなのだ。

\section{「ポロポロ」におけるホラーと小実昌の慰霊}

本節では、保坂が注目していた小実昌のホラー映画に対する恐怖心と「ポロポロ」の物語構造の関係について論じていく。

私は、まず、「ポロポロ」が最初の短編として置かれていることが、「ポロポロ」がなぜある種の幽霊譚となっているのかの理由を説明したい。その議論をするためには、軍隊連作短編がどのような構成になっているのかを明らかにして、そこに「ポロポロ」がどのように当てはまるかを説明する必要がある。以下では、それについて述べてたい。

最初に、短編集に収められている作品を掲載順に並べてその初出年号を確認しておく。

\textgt{ 短編集『ポロポロ』(中央公論社、1979)所収作品(括弧内は初出情報)}
\begin{itemize}
  \item ポロポロ(『海』12月号、26-39頁、1977)
  \item 北川はぼくに(『海』3月号、20-33頁、1978)
  \item 岩塩の袋(『海』6月号、22-37頁、1978)
  \item 魚撃ち(『海』9月号、10-24頁、1978)
  \item 鏡の顔(『海』1月号、87-100頁、1978)
  \item 寝台の穴(『海』4月号、34-47頁、1978)
  \item 大尾のこと(『海』5月号、160-179頁、1979)
\end{itemize}

語り手以外に主要な登場人物がいない作品は「鏡の顔」と「寝台の穴」のみである。前者は、自身が鏡を見ているときに父親そっくりになっていたことに驚くというもので、後者は赤痢になった時の経験を踏まえた野戦病院での療養記録のようなものである。

そのほかの作品に共通しているのは、「北川」や「大尾」といったように、軍隊に所属している他の誰かについて語り手が死んでいった様子やあるいは巻き起こした騒動について語っていくものである。以下では、これらの作品の特徴を見ていきたい。

「北川はぼくに」では、語り手が、昭和19年12月24日満19歳で山口の連隊に入営したとされている。ちょうど、「ポロポロ」から3年後から物語が始まっている。こうした連作には次に見るように、語り手は現実の状況に対して常に距離をとった態度を示している。

\begin{quote}
  ぼんやりとしていたわけでもない。へえ、負けたのか、とごくふつうにおもっただけだ。これも、ぼくがだれかとスモウをとって、負けたのではない。戦争に負けたということなのか、とおもったにすぎない。諦観的というのでもない。とにかく、なんともおもわなかった\footnote{田中小実昌「北川はぼくに」『昭和文学全集』、第31巻、小学館、1988年、477頁。}。
\end{quote}

また、小実昌自身が従軍経験者であることを踏まえて、同世代の初年兵たちが戦後になって敗戦を悔しがったということに対する違和感の表明も、当事者による批判ありながら客観的に述べられている。
\begin{quote}
  よけいなことだが、あのころは、戦争に負けたことへのくやしさ、なさけなさといったものは、上級の兵隊と初年兵とではうんとちがっていたはずだが、当時の初年兵に、今たずねたら、上級の兵隊だった者と、あまりかわらないことを言うのではないか。それにぶつかったとき、自分が感じたこと、おもったことが、だんだんにかたちを変えて、つまりは、世間の規格どおりみたいになるのだろう。これはふしぎなことだが、世間ではあまりふしぎにおもってないようだ。ま、そんなふうだから、こんなことにもなるのか\footnote{前掲書、478頁。}。
\end{quote}
こうした感覚は「魚撃ち」にも見られる。
\begin{quote}
  また、敗戦を終戦と言う言葉でゴマかしたという論議が、後になってでたようだが、ぼくカンケイない\footnote{田中小実昌「魚撃ち」『昭和文学全集』、第31巻、小学館、1988年、483頁。}。
\end{quote}
こうした客観的な語りは、プラグマティズムのクリシェである、悲しい感情が涙を流させるのではなくて、涙を流す生理機能が悲しみの感情を生み出すというのと深く関わっている。『ポロポロ』以後、小実昌は、宗教心と宗教を分けるように、感情と生理を峻別するような態度を、哲学小説や対談で取るようになるが、実は「大尾のこと」の中で語り手の態度として現れている。

中国で終戦を迎えた語り手が、現地の人々との交流を描く一節での言葉である。語り手は、中国人は成人であれば敵国であった自分たちを今や商売相手としてしか見なさず物を売りに来るのに対して、現実の子どもたちは悪口らしきことを言いながら投石してくることについて、語り手は次のように述べている。

\begin{quote}
  憎む、というような、たいへん直接的な感情でも、やはり学習しないと、その感情があらわれてこないのではないか\footnote{田中小実昌「大尾のこと」『昭和文学全集』、第31巻、小学館、1988年、483頁。}。
\end{quote}

つまり、現地の大人たちは生活のために実利を得ようとして憎しみから距離を取ることができるのに対して、子どもたちはそのまま日本兵に対する憎しみを内面化して感情的な行動をとると述べている。これは、語り手が戦友の悲惨な死を目撃しつつも常に客観的であることが物語の中で入れ子状に語られているとも言える。語り手は戦友の死を悼むのだが、それは慰霊のような形では決してありえない。語り手はそのような「学習」をしなかったからである。私の考えでは、軍隊連作小説以降の小実昌作品の語り手が常に感情から距離を取るような態度を示し、戦後の同世代の初年兵に対する批判的な言説を行なっていることは関係している。すなわち、戦友の喪失や過ちといったものに対する鎮魂の行為に対して、どのようなアプローチが可能なのか、という問いを小実昌は立て、それが感傷を伴った一般的な慰霊ではありえないということを示しているのである。「ポロポロ」でなぜ死んだ祖父という幽霊が登場したのかの理由は、まさにこの語り手の客観性にあるのだ。

以上を踏まえて、「ポロポロ」について論じる。まず、あらすじを確認する。

昭和16年12月のある日、語り手「ぼく」の父方の祖父の命日ということで、僕はせんべいを買いに遠出する。その帰り道に背格好が、祈祷会でいつも「ポロポロ」といつも言っている一木さんにそっくりな人を見かけ、挨拶して横を通り過ぎる。一木さんが玄関から入ってこれるように戸を開けておいた「ぼく」だったが、今日の祈祷している部屋にもう一木さんがいた。さらに、玄関の戸は音もなく閉まっていた。「ぼく」はそのことを父に話した結果、死んだ祖父だったということになり、「ぼく」はそれを「ポロポロ」と受け入れることにする。

語り手が石段で出会った人の正体は、作品中の言及や戦争期であることを踏まえると、霊的な現象でなければ、特高であると自然に推察できる。しかし、疑念が残るのは、保坂が指摘していたように、扉が音もなくしまっている所に固執する箇所である。実際、主人公たちが祈っている床の間は玄関から続く廊下にあるのだから、明らかに扉の開け閉めの音を聞くことができるような場所として設定されている\footnote{『アメン父』によれば、実際に床の間を勉強部屋として使用していたことが窺える。}。よって、この見かけた人影について、語り手が「人魂」の目撃談を紹介していた伏線をふまえ、あえて「人魂」だと本論では解釈する。それでは、なぜ連作短編でもっとも注目される人物は人魂なのだろうか。それについて以下で考察する。

「ポロポロ」のラストシーンで祖父の「人魂」が出現したと結論づける重要な伏線は、冒頭の見渡せるはずの階段に突然現れた人影などがあるが、畑をめぐる挿話にもそれが現れている。教会の段々畑の一番下には信者ではない人々の墓があった。ある日、人夫たちによってその下の段にあった別の土地の墓が勝手に移されてしまう。その後、まだ教会の庭の植木を見ていた一家の母が人魂を目撃する\footnote{田中小実昌「ポロポロ」『中央公論』、第94年第11月号、中央公論社、1979年、366頁。}。このことから、冒頭の描写で御影石の石段を上る主人公が段々畑に入って階段を登るときにあるものを目撃したことが省略されていると考えられる。「子供のぼくの目にもどさくさまぎれという感じ\footnote{前掲書、366頁。}」で移動させられた、おそらく整序されていないと想像される墓の前を通っていたことが意図的に伏せられいるかのようである。こうした夜中に墓所の横を通り過ぎるという一読すれば推察できる様子が冒頭で描かれていないのは、保坂が指摘しているようにホラーな状況を喚起させることに対して抑圧的なためかもしれない。いずれにせよ、この挿話を踏まえて冒頭を再度解釈し直すと、「冬のはじめのしらじらしい月夜\footnote{前掲書、360頁。}」にせんべいか何かを買いに出た主人公は、その帰り道に、乱雑に並べられた暮石の前の道を通り過ぎて、祖父の「人魂」を見かけ、一木さんと取り違えた、となる\footnote{ちなみに、この語り手はウェイン・ブースが『フィクションの修辞学』で提唱した「信頼できない語り手」である。そのことは他の点からも指摘できる。語り手が別の箇所で述べているように、「山の中腹の木立のなかの日本家屋の教会では、礼拝の時間がきて、父はポロポロはじめており、時間におくれて、坂をのぼり、畑のわきの野道をやってくる人たちが、あるきながら、ポロポロやってるなんてことは、いつものことだった」(前掲書、368頁)のだから、家に帰って「祈祷会」が始まった時点で黙っている歩いているのは信者ではありえず、それが一木という熱心な信者ではなおさらありえないのに気づけたのにもかかわらず、そのことに言及していない。}。

ところで、そもそもなぜ人夫たちは教会の土地に勝手に墓を移して良いと考えたのだろうか。これは、隣の農家の「オジさん」が少しずつ私道を自分の土地にしていったことを踏まえると、山村の共同体から語り手の所属する共同体である教会が排除されていたと考えられる。もともと街にあった教会が山腹に移動したこと、その移動先でも排除されていることといったように常に社会から排除される語り手の姿が浮き彫りになっている。さらに、連作でも注目されるのはいつも何らかの形で排除された人々である。日本兵を殺してしまい、自分にだけ打ち上げ場話をするが戦後に出会った後その消息が不明である北川やコレラと人の嫌がる仕事を率先してこなしていたために周りから軽蔑されて最後は栄養失調で死んだ大尾と交流することは、語り手自身の物語と言える「ポロポロ」においてあらかじめ書き込まれていたのである。先にも挙げたように、隣家の農夫が私道を強引に私有地化した箇所で、語り手は「そんな百姓のオジさんはわるい人でも、不正直と言われるような人でもなく、勤勉な、ごくふつうの人なのだ\footnote{前掲書、367頁。}と語っているものの、私道の簒奪が「ごくふつう」のこととされるのは、主人公が自らの社会的排除を内面化していることを示している。先ほど述べたプラグマティックな文言になぞらえるのであれば、排除を学習していないので、それを感情的に表現できないのだ。

しかし、そうした排除の構造こそ、「人魂」と出会うことを可能にしていると考えられる。というのも、私の考えでは、語り手が石段で出会った人物が特定の登場人物であるというのではなくて、排除されたものの象徴として連作短編の登場人物の誰もがあの人魂でありえるということなのだ。それについて以下で見ていく。

田中の作品は、例えば、「ミミのこと」や「浪曲師朝日丸の話」に代表されるように、一人の人物について巡る話が多いように思われる。しかし、「ミミのこと」はミミに焦点を当てつつも、語り手の周囲との人間関係が描かれる。「浪曲師朝日丸の話」は、元子が語り手の話を記者に教えて、軍隊時代の友人の朝日丸が週刊誌に美談として取り上げられたことに触れつつ、実際は引き取った震災孤児たちをストリッパーにして儲けたうえ、全員に子を産ませているという事実があるために、その事態の収集を不安に思うという構成になっている。それは、例えば、「ポロポロ」で主人公が父や母について十分に記述して自分との関係を考察したり、一木がポロポロと言っているのを分析する\footnote{「ポロポロは宗教経験でさえない。経験は身につき、残るが、ポロポロはのこらない。だから、たえず、ポロポロを受けなくてはならない」。前掲書、370頁。}のと大きく異なっている。「ポロポロ」の語り手は「ポロポロ」が何であるかを饒舌に語るが、取り違えられた祖父や一木がどんな人物であったかについてはほとんど読者に教えてくれない\footnote{「ポロポロ」から知ることができるのは、その日が父方の祖父の方の記念日だったことだけである。また、『アメン父』を参照しても、結局この日が何日かは不明である。}。また、一木について、ポロポロ言っていること以外は読者は何も知ることができない。その上、その「人魂」が「ソフトをかぶり、二重\bou{まわし}を着て\footnote{傍点原文ママ。前掲書、360頁。}」いるのは、「そのころの大人はみんなそうだが、一木さんもソフトをかぶり、冬になると、二重\bou{まわし}を着ていた\footnote{傍点原文ママ。前掲書、361頁。}」とあるように、「人魂」はそもそも多くの人に当てはまりえた。世相を鑑みれば、それが特高であったとも考えた。こうした誰にでも当てはまりえたという類似性に焦点を当てることで、連作短編の登場人物たちの誰もが人魂でありえることの理由はさらに明白なものとなる。以下で、詳しく見てみよう。

語り手と人魂を除いた最初の登場人物は一木であった。この一木に比べると、母についての情報は語り手によって多く与えられる\footnote{前掲書、368-369頁。}。重要な点としては、一木と同様に足に障碍を抱えている点である。一木は、「リューマチかなにかで、足が不自由で、その夜、石段の上にいた人の足もとがおぼつかなく見えた\footnote{前掲書、365頁。}」という。語り手の母は、「足に骨髄炎かなんかをおこして、手術をし、ところが、その手術のときにバイキンがはいって(中略)、二度手術をし、(中略)片足の膝のまがらないびっこになってしまった\footnote{前掲書、368頁。}」。このように、二者は足に不具合を抱えており、かつ、それがどちらの足かわからないという点で共通している\footnote{推測の域をでないが、語り手が母が祈祷の際に投げ出している不具の足に座布団を被せている行為について、「だれか、やはり足がわるいひとがそうしてるのを見て、母は真似をしたのか?」(前掲書、368頁)という疑問から連想を重ねている。真似しているかどうかは定かでないものの、少なくとも「人魂」で問題とされている類似性が誰かの真似であるかもしれないという言い方によって書き込まれている。}。さらに、母について語り手が「母には自分の肉体のことなどは、どうでもいいようなところがあった\footnote{前掲書、368頁。}」と表現していることは、単に障碍に無頓着であるということではなくて、具体的な身体に依拠しないでいるとも読み替えることができ、それは、あたかも「人魂」のような精神的存在を志向しているかのようである。加えてもう一つ、一木との共通点を挙げておくと、母が誰かと会話する場面は作中に存在していない。

作中の中で唯一語り手の会話が記述されているのは、父とのもので、「じゃ、おじいさんだ。記念日だから、おじいさんがきたんだよ\footnote{前掲書、373頁。}」である。これ以外は会話がないこの短編において語り手の父のこの発言は作中における例外である。このことから、父は特殊な存在であり、人魂の代替になるような類似性を共通しているわけではないように重われる。しかし、父の過去の話はすべて母から伝えられるのに注目すると、会話をしているからといって例外的な存在ではありえないことは明らかである\footnote{「だが、そのはなしを、父の口からは一度もきいたことはない。みんな、母がはなしてくれたことだ。」など。前掲書、369頁。}。結局のところ、語り手の話は母からの伝聞であり、父についても私たちはそれが作中においてさえ事実かどうか確かめることができないようになっている。一木や母の不具の類似性の他にも、こうした伝聞のみで事実が不明であること(そもそも、人魂を見たことも、人魂が人家の屋根を伝っていたというのも伝聞だった)もまた、「ポロポロ」の登場人物に特徴的なことだろう\footnote{ここで触れていない妹について補足しておく。現実の田中小実昌の妹は、この家から出ることなく牧師の夫を迎えて教会を維持することになる。ただし、作中における妹についての描写は、他の田中作品における語り手と近い立場にある登場人物についてのものと比べても、例外的に少ない。}。

一旦、結論をのべる。まず、主人公の目撃した「人魂」はそもそも「祖父」ではない。より正確にいうと、祖父であっても、そうでなくとも良い。ただし、語り手の置かれている立場と同様、排除されている者であるならば、誰もが「人魂」に当てはまるのである。なぜなら、作中の誰もが多かれ少なかれ、それぞれが「人魂」を経由して類似しているのだ。そして、これが連作短編の最初であったことを踏まえると、この「人魂」は連作で焦点の当てられる登場人物たちのことでもあるのだ。「人魂」は語り手が用意した鋳型と言える。この仮説について、連作短編の位置付けという観点から、検討し直してみよう。

そもそも、「ポロポロ」は、「ポロポロ」の以外の全てが入隊以後の話である。よって、「ポロポロ」が最初に配置されているのにはそれなりの理由があったと考えるべきである。「ポロポロ」では、語り手が自身の誕生日を「関東大震災から二年後、大正十四年(一九二五年)\footnote{前掲書、371頁。}」としている。関東大震災は父の物語を語るうえで触れられていた。父のポロポロが始まったのは、この年である。作中の描写からは、晩産や子供の誕生がきっかけにポロポロが始まったように見える。加えて、歴史的な事実を述べれば、1925年は治安維持法が制定された年である。治安維持法は、思想検閲や結社の制限であり、キリスト教徒であった人々の不安を煽ったのは想像に難くない。さらに、父は震災でデマに煽られて「朝鮮人」を殺そうとする人々を目の当たりにしている。こうしたことが信仰の不安に繋がっていったと想像できる。しかし、「ポロポロ」の時系列を整理すると、それ以外の解釈の可能性が生じる。

「太平洋戦争がはじまった十二月八日に近い夜だったかもしれない\footnote{前掲書、364頁。}」とあるように、作品自体は開戦の瞬間の物語である。すると、「ポロポロ」には治安維持法が制定されてから太平洋戦争が開戦するまでの範囲を物語中に見いだすことができる。以上から、作品で「人魂」が祖父のものであると一応の決着を見たことも別の意味があるように思われる。

例えば、明治18年生まれの父のその父は、江戸時代に生まれており、時代の転換期の象徴である。すると、太平洋戦争の開戦の時期が物語内の時間であり、そこに時代の転換期の象徴が「人魂」として表現されるのは、これが時代の転換であったことを暗示しているのである。「ポロポロ」と名指されている信仰にまつわる問題は、語り手が「宗教経験でさえない\footnote{前掲書、370頁。}」と明確に述べているように、確かに宗教経験の問題ではない。それは、戦争体験についての問題となるのだ。連作のなかで唯一軍隊所属の登場人物が出てこない「ポロポロ」は市井の人の戦争体験であり、それ以後の連作は軍隊での戦争体験なのである。

「ポロポロ」とそのほかの短編の関係性を理解したうえで、『ポロポロ』所収の軍隊小説のうちある一人の人物に焦点を当てているものと「ポロポロ」を比較して考えてみると、「ポロポロ」では焦点が当たっていた人物が「人魂」であったことがわかる。すでにみたように、その「人魂」は誰かと共通点を常に持っているものであり、ある人物であるかどうかも限定はできない存在として描かれている。よって北川や大尾の鋳型として「人魂」が描かていたのだ。

しかし、こうした考えは、「大尾のこと」で提起されている物語への懐疑とどうつながっているのだろうか。最後にその点について論じてみたい。

本論でのこれまでの「ポロポロ」の読解では、登場人物の鋳型としての「人魂」という結論を引き出した。以下で、後期小実昌の嚆矢とされる『ポロポロ』以後、登場人物というテーマはどのように扱われているのかを見ていくことで、物語と登場人物をめぐる小実昌の思想と、「人魂」というホラーを通じて登場人物が表象されるのか論じる。

後期小実昌の代表的な哲学小説「カント節」の語り手は、カントの『純粋理性批判』をバスの中で読みながらその文章を引用しつつ、連想広げていく。その末尾で、小実昌自身が対談などで時折述べていた哲学者は神について語る一方でなぜイエスについて語らないのかという疑問をここでも提示している\footnote{田中小実昌「カント節」『昭和文学全集』、第31巻、小学館、1988年、516頁。}。ところで、イエスは信仰の対象であると同時に、新約聖書で中心となる登場人物である。すると、田中の疑問は、どうして登場人物については語らないのかと言い換えることができる。そして、神については、「大尾のこと」のことで「はじめと、おわりのある物語\footnote{田中小実昌「大尾のこと」『昭和文学全集』、第31巻、小学館、1988年、502頁。}」について、「ひとのはじめとおわりに関与するなど、神のすることではないか\footnote{同書。}」と述べていることから、物語に関わることだと言い換えられる。では、哲学者は何に言い換えられるだろうか。これは、物語の語り手と言ってよい。「カント節」の最後の一文は、「これらの哲学者にとっては、神もイエスもひとつのものなのか\footnote{田中小実昌「カント節」『昭和文学全集』、第31巻、小学館、1988年、516頁。}」となっており、物語と登場人物を1つにするのは明らかに語り手のことである。

よって、『ポロポロ』以後の後期小実昌にとって問題だったのは、どうして語り手は物語について語るのに登場人物については語らないのか、と言い換えることができる。すると、先行研究で述べられていた事実を物語ることへの疑義という後期小実昌のテーマは、そのままイエスと神の問題であったと言える。

とはいえ、それが単なる宗教の問題ではないことは、すでに注目した「カント節」の最後の一節の少し前の箇所から指摘できる。小実昌はカントが「かのように」と留保をつけながら論述している箇所を3つ引用している\footnote{同書。}。物語と登場人物についての考察は、実は「かのように」という虚構性に注目した文章のすぐ後に配置されているのだ。すなわち、これは後期小実昌のテーマであった、事実と事実を物語にすることで生じる「ウソ」への倫理的な警戒とも関わっているのである。そして、この事実と物語の「ウソ」について、小実昌が自覚的な発言をしたのは、すでに引用したように、富岡が『アメン父』と比較して、「ポロポロ」の虚構性を指摘したのに応じた時であった。

では、なぜそもそも小実昌にとって物語の虚構性を意識し続けながらも物語を描く必要があったのだろうか。それには2つの理由があると考えられる。それはホラーと慰霊という2つのキーワードによって考えることができる。

保坂が指摘したいたように、「ポロポロ」で語り手がしきりに妹に対して扉を閉めていないか確認しているという描写は、語り手がそれが神秘的な現象であることを強調するとともに、心霊現象に対して強い恐怖を覚えているために神経質に尋ねてしまっているという両方の可能性が考えられる。他にも、ホラーに対する小実昌の意識がわかるのは、語り手の話法においてである。「ポロポロ」は、移しかえられた墓の横を真冬の夜に通り抜けていくというシチュエーションや集会が始まっている時点で、すれ違った人物が信者ではありえないといったように、文中で与えられている情報を総合すると語り手が語っていないことのために、かえってその内容を想像で埋めるほかないようなスタイルで書かれた、ある種の黙説法が用いられている。この黙説法によって小実昌は常にその対象について語っていながらその対象について具体的に語ってはいないという空白を生み出すことになる。実は、この空白こそ小実昌にとってのホラーであり、登場人物のことなのだ。小実昌が、事実が虚構になってしまう物語について語るのではなくて登場人物について語ることが重要である倫理的態度をとり、「ポロポロ」という異言を用いることで言葉を否定しているかのようにも思われるのにはこうした小実昌のホラーが存在している。小実昌にとって登場人物とはある種の幽霊なのである。

以上のように、鋳型としての登場人物が「人魂」だった理由は、こうしたホラーと登場人物の関係が小実昌の中で密接に結びついている。『ポロポロ』以前の短編が所収されている『香具師の旅』が同年に刊行されているのにもかかわらず大きく異なる性格を持ち合わせている理由はそこに由来している。すなわち、『香具師の旅』は虚構の人物についての虚構の物語であるのに対して、『ポロポロ』に所収されている軍隊小説は、太平洋戦争(と日中戦争)と死んだ(あるいは味方の兵を殺してしまった)兵士という事実に基づく虚構の物語なのである。小実昌が登場人物の鋳型として幽霊を選び、それがホラーに結びついてしまうことこそ、小実昌が物語を作るうえで常に問題にしたことだった。

そして、こうした登場人物の捉え方はある意味で小実昌独特の慰霊、というよりも、戦後の戦争文学と慰霊をめぐる言説に対する小実昌のアプローチだったと考えられる。『ポロポロ』の連作短編の中で幾度か初等兵たちが戦後になって上等兵たちと同じように嘯いていたことへの批判的言及や終戦か敗戦かという議論が語り手にとってどうでもよいとされていたのは、戦争を体験し、餓死しかけた者としての、終戦論争や慰霊論争に対する抵抗だったかもしれない。イエスが物語の中では復活しえたように、戦争の死者は物語の中では復活しえるのだ。小実昌が『アメン父』で宗教者としての父にこだわったのは、神のみがなせるはずのその業とどのように向き合えば良いか迷ったからもしれない。

ここまで「ポロポロ」の読解を通じて、そこで出された登場人物の鋳型としての人魂が、黙説法という語りの特徴を通じて、小実昌のホラー的感性を明らかにしつつ、そうした幽霊のようなものに依拠して表現することが独自の慰霊だったのではないかという可能性を示した。先行研究の多くが見逃していたのは、物語に対する懐疑心や異言を取り上げるということが小説の限界を指し示すといった形式的な問題ではなく、戦争といった災禍を生き延びた者がどのように死んだ者の代わりに言葉を継げるのか、という普遍的な倫理の問いがその形式にとり憑いていることであった。

\end{document}

